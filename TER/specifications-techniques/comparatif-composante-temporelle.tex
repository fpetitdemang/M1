\documentclass[10pt,a4paper]{article}
\usepackage[latin1]{inputenc}
\usepackage{amsmath}
\usepackage{amsfonts}
\usepackage{amssymb}
\author{Franck Petitdemange}
\title{Comparatif d'Api pour la composante temporelle}
\begin{document}
\subsection*{http://timeline.verite.co/}
C'est un outil très ergonomique.Cependant pas adapté à nos besoin car il n'est pas prévu pour afficher un gros volume d'informations. Cependant on peux s'inspirer des différentes fonctionnalités visuelle, notamment : 
/*** Prévoir une petit analyse des fonctionnalités intéressantes **/

\subsection*{http://flowingmedia.com/timeflow.html}
TimeFlow est un projet open source développé initialement pour permettre a des journalistes d'avoir un outil d'analyse de données dans le temps. Une des contraintes intéressante du projet et qu'il a été développé pour afficher des centaines de données.

\subsection{http://www.simile-widgets.org/timeline/}
\end{document}